%%% Løsningskoncepter til problemet
%% Herefter en kravmatrix til vurdering af løsningskoncepterne

%%

\subsection{Krav til produktet}
For at komme frem til et endeligt produkt, som kan løse problemet, opsættes en kravspecifikation som ses nedenfor.

Funktionelle krav:
\begin{itemize}
    \item Mulighed for at tilkalde nødhjælp
    \item Hjælp til førstehjælper
\end{itemize}

Kvalitetskrav:
\begin{itemize}
    \item Enkel og intuitiv brugergrænseflade
    \item Offline-funktion (Produktet skal kunne bruges, selvom der ikke er internetforbindelse)
    \item Hurtig og effektiv
\end{itemize}

Brugerkrav:
\begin{itemize}
    \item Produktet skal have mulighed for at understøtte flere sprog
\end{itemize}

\subsection{Løsningskoncepter}
\subsubsection{Koncept 1}
En app, som kan afhjælpe til at tilkalde nødhjælp i situationer, hvor dette er nødvendigt. Appen vil endvidere kunne hjælpe førstehjælperne med at komme i gang med at hjælpe den person, som har kaldt nødhjælp. Appen kan nemt dele brugerens lokation med relavante personer, dette gøres enten ved koordinater eller ved google maps.

\subsubsection{Koncept 2}

\begin{table}[H]
    \begin{tabular}{|l|l|l|l|}
        \hline
        \textbf{Løsning} & \textbf{Funktionelle (vægt 5)} & \textbf{Kvalitetskrav (vægt 4)} & \textbf{Brugerkrav (vægt 3)} \\
        \hline
        \textbf{alarm-app} &  &  &  \\
        \hline
        \textbf{Nød-"radio"} &  &  &  \\
        \hline
    \end{tabular}
\end{table}


\subsection{Konkurrentanalyse}

\subsection{Endeligt produkt}
Ud fra kravmatrix og konkurrentanalyse, vælges det endelige produkt til at være alarm-appen.


Vi har valgt flg. koncept til løsning af problemet:

%% Konkurrentanalyse

%% Ny kravspecifikation

%% Hvilke løsningsforslag er der til det tekniske problem?

%% Vurdering afløsningsforslagene ved hjælp af kravmatrix

%% konklusion: hvilket produkt har i valgt?



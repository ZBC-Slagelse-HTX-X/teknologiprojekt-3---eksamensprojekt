Vi lever i et højere teknologisk samfund, hvor mange af os benytter teknologi på daglige basis, men I nutidens tilfælde af krig er der rigtigt stor risiko for hacker angreb, som ikke bare svækker militære styrker, men også vores allesammen hverdag.
Det leder os frem til følgende overordnede spørgsmål, vi vil besvare i rapporten:

\begin{itemize}
    \item Hvad er årsagerne til, og konsekvenserne af, teknologiske angreb under krig - og hvad kan vi gøre ved det?
\end{itemize}

For at kunne besvare dette spørgsmål, må vi først undersøge følgende som ses neden for; Disse er opstillet udfra problemtræet \ref{fig:problemtræ}:
\begin{itemize}
    \item Hvordan skal vi kunne gøre kommunikation nemmere under en krig
    \item Hvad skal man kunne gøre i tilfælde af blackout
    \item Hvordan skal man ku tilkalde hjælp og backup under krig
    \item Hvordan skal samfundet hænge sammen forhold til de teknologiske blackouts.
\end{itemize}

Ovenstående spørgsmål vil blive besvaret i problemanalysen
Problemanalyse

Årsagerne til teknologiske angreb under krig
En af de største årsager til teknologiske angreb under krig er den stigende afhængighed af teknologi i militære operationer.
herefter kan de også måle deres angreb mod civil infrastruktur. som gir fjenden muligheden for, at målrette deres angreb mod kritisk infrastruktur og kommunikationssystemer. konsekvenserne kan være, at Hospitaler og redningskøretøjer ikke kan kommunikere med hinanden, som vil resultere at civile mennesker med kritiske tilstande ikke kan få hjælp i tide.
ikke vil kunne få hjælp i tide. og herefter også folk som er hospitaliseret som har brug for højere teknologiske værktøjer til at kunne overleve, heller ikke vil være istand til at få hjælp. kommunikation mellem lægerne vil også miste effektivitet, grunden manglen på kommunikation mellem hospital sektioner.
